\documentclass[12pt]{article}
\usepackage[utf8]{inputenc}
\usepackage[russian]{babel}
\usepackage[left=2.5cm,right=1cm,
top=2cm,bottom=2cm,bindingoffset=0cm]{geometry}
\usepackage{amsmath,amsfonts, amssymb}
\usepackage{graphicx}
\usepackage{listings}
\usepackage{xcolor, lipsum}
\lstloadlanguages{MATLAB}

\definecolor{back}{HTML}{FBFDFF}
\definecolor{board}{RGB}{77, 101, 104}
\definecolor{keywords}{RGB}{166, 38, 164}

\lstset{
	language=Matlab,
	commentstyle=\color[rgb]{0.8, 0.5, 0.3},
	showstringspaces=false,
	keywordstyle=\color{keywords},
	identifierstyle=\color{black},
	backgroundcolor=\color{back},
	extendedchars=\true,
	keepspaces=true,
}

\usepackage{caption}
\DeclareCaptionFont{white}{\color{white}}
\DeclareCaptionFormat{listing}{\colorbox{board}{\parbox{\textwidth}{\hspace{15pt}#1#2#3}}}
\captionsetup[lstlisting]{format=listing,labelfont=white,textfont=white, singlelinecheck=false, margin=0pt, font={bf,footnotesize}}

\newtheorem{theorem}{Задача}

\begin{document}
\begin{titlepage}
	\begin{center}
	\large {
	ФГБОУ ВПО\\
	ПЕТЕРБУРГСКИЙ\\
	ГОСУДАРСТВЕННЫЙ УНИВЕРСИТЕТ ПУТЕЙ СООБЩЕНИЯ\\
	ИМПЕРАТОРА АЛЕКСАНДРА I}
	\vspace{2cm}
			
	Кафедра "Информационные и вычислительные системы"\\
	\bigskip
	\bigskip
	\bigskip
	\bigskip          
			
	\large
	ОТЧЕТ\\
	по лабораторной работе №1\\
	{\LARGE\textbf{Поиск подстроки в строке}}
	\end{center}
		
	\newlength{\ML}
	\settowidth{\ML}{дент}
	\bigskip
	\bigskip
	\bigskip
	\bigskip
	\bigskip
	\bigskip
		
	\begin{flushleft}
		Выполнил студент \hspace{4.0cm} $\underset{\text{(подпись)}}{\underline{\hspace*{2.5cm}}}$ \hfill
		Зайцев Л.А.\\
		Группа ИВБ-811
	\end{flushleft}
		
	\bigskip
	\begin{flushleft}
		Отчет принял \hspace{\ML}\hspace{4.0cm} $\underset{\text{(подпись)}}{\underline{\hspace*{2.5cm}}}$ \hfill Баушев А.Н.
	\end{flushleft}
	
	\vfill
	\begin{center}
		Санкт-Петербург, 2020 г.
	\end{center}
\end{titlepage}

\section{Введение}
\textbf{Поиск подстроки в строке} (англ. \textit{String searching algorithm}) -  класс алгоритмов над строками, которые позволяют найти паттерн (\textit{pattern}) в тексте (\textit{str}).
\begin{theorem} \label{task1}
	Дана строка $ str[1 .. n] $ и паттерн $ pattern[1 .. m] $ такие, что $ n >= m $ и элементы этих строк — символы из конечного алфавита $ \sum $. Требуется проверить, входит ли $ pattern $ в $ str $.
\end{theorem}
Эти алгоритмы подразделяются на несколько групп:
\begin{itemize}
	\item \textbf{Сравнение — «чёрный ящик»}\\
	Во всех этих алгоритмах сравнение строк является «чёрным ящиком». К этим алгоритмам относится \textbf{примитивный алгоритм}.
	\item \textbf{Основанные на сравнении с начала}\\
	Это семейство алгоритмов страдает невысокой скоростью на «хороших» данных, что компенсируется отсутствием регрессии на «плохих». К этим алгоритмам относятся \textbf{алгоритм Рабина-Карпа} и \textbf{алгоритм Кнута-Морриса-Пратта}.
	\item \textbf{Основанные на сравнении с конца}\\
	Сравнение строк друг с другом проводится справа налево.
	\item \textbf{Проводящие сравнение в необычном порядке}
\end{itemize}
\section{Описание алгоритмов и их реализация}
\subsection{Примитивный алгоритм}
\subsubsection{Описание алгоритма}
В примитивном алгоритме поиск всех допустимых сдвигов производится с помощью цикла, в котором проверяется условие \textit{str[i .. i + m - 1] = pattern} для каждого из \textit{n - m + 1} возможных значений i.
\subsubsection{Код программы}
\begin{lstlisting}[language={Matlab}, caption={Примитивный алгоритм поиска подстроки в строке}, label={Script}]
function [index] = Pos(str, pattern)

n = strlength(str);
m = strlength(pattern);
for i = 1 : n - m + 1
  if str(i : i + m - 1) == pattern
    index = i;
    return;
  end
end

index = -1;
end % End of 'Pos.m' function
\end{lstlisting}

\subsubsection{Сложность алгоритма}
O((n - m) * m)

\subsubsection{Результаты работы}
m = [5, 10, 20] (красный, зеленый, синий)
\begin{figure}[h]
	\centering
	\includegraphics[width=0.8\linewidth]{Pos.jpg}
	\caption{Результат работы примитивного алгоритма}
\end{figure}
\subsection{Алгоритм Рабина-Карпа}
\subsubsection{Описание алгоритма}
Для реализации данного алгоритма нам понадобится хэш-функция. Воспользуемся полниномиальным хэшем:
\begin{equation}
h = hash(s[1 .. n]) = (p^{n}s_{1} + p^{n - 1}s_{2} + ... + p^{0}s_{n}) \: mod \: q,
\end{equation} где \textit{p} и \textit{q} заранее заданные числа.\\
Отсюда следует, что:\\
\begin{equation} \label{eq2}
h[i + 1] = h[i] \cdot p + s[i]
\end{equation}
Исходя из (\ref{eq2}) можно сделать вывод, что для того, чтобы пересчитать хэш-код за O(1) нам необходимо воспользоваться формулой: \\
\begin{equation}
hash(s[i .. i + m - 1]) = (p \cdot hash(s[i - 1 .. i + m - 2]) - p^{m}s[i - 1] + s[i + m - 1]) \: mod \: q
\end{equation}
Значения для \textit{p} и \textit{q} следует выбрать таким образом, чтобы уменьшить вероятность коллизий. В частности, мы хотим минимизировать количество таких остатков, которые не могут быть хэшем никакой строки. Тогда возьмём взаимнопростые числа.\\
Алгоритм:
\begin{enumerate}
	\item Вычисляем значение хэш-кода для паттерна и для первых \textit{m} символов в тексте.
	\item Сравниваем хэш код для паттерна и текущей подстроки. Если они равны, то сохраняем индекс и выходим.
	\item Считаем значение хэш-кода для следующей подстроки, переходи к шагу 2.
\end{enumerate}
\subsubsection{Код программы}
\begin{lstlisting}[language={Matlab}, caption={Алгоритм Рабина-Карпа}, label={Script}]
function [index] = RabinKarp(str, pattern)
n = strlength(str);
m = strlength(pattern);
q = 433494437;
p = 29;
h = 1;

% Evalute p^m
for i = 2 : m + 1
  h = ModuloMult(h, p, q);
end
hStr = 0;
hPattern = 0;
% Evalute hash value for pattern string and the first window
for i = 1 : m
  hPattern = ModuloAdd(pattern(i), ModuloMult(p, hPattern, q), q);
  hStr = ModuloAdd(str(i), ModuloMult(p, hStr, q), q);
end

if hStr == hPattern
  index = 1;
  return;
end

for i = 2 : n - m + 1   
  hStr = ModuloAdd(ModuloMult(p, hStr, q), ModuloAdd(-ModuloMult(h, str(i - 1), q), str(i + m - 1), q), q);

  if hStr < 0
    hStr = hStr + q;
  end

  if hStr == hPattern
    index = i;
    for j = index : index + m - 1
      if str(j) ~= pattern(j - index + 1)
        break;
      end
    end
    return;
  end
end
index = -1;
end % End of 'RabinKarp' function

function res = ModuloAdd(x, y, q)
res = rem(rem(x, q) + rem(y, q), q);
end % End of 'ModuloAdd' function

function res = ModuloMult(x, y, q)
res = rem(rem(x, q) * rem(y, q), q);
end % End of 'ModuloMult' function
\end{lstlisting}
\subsubsection{Сложность алгоритма}
O(n) в среднем, в худшем O(nm)
\subsubsection{Результаты работы}
m = [5, 10, 20] (красный, зеленый, синий)
\begin{figure}[h]
	\centering
	\includegraphics[width=0.8\linewidth]{RK.jpg}
	\caption{Результат работы алгоритма Рабина-Карпа}
\end{figure}
\vfill

\subsection{Алгоритм Кнута-Морриса-Пратта}
\subsubsection{Описание алгоритма}
Дана цепочка T и образец pattern. Требуется найти все позиции, начиная с которых pattern входит в T. 
Построим строку S = pattern\#str, где \# — любой символ, не входящий в алфавит pattern и str. Посчитаем на ней значение префикс-функции p. Заметим, что по определению префикс-функции при i > |pattern| и p[i] = |pattern| подстроки длины pattern, начинающиеся с позиций 0 и i - |pattern| + 1, совпадают. Если в какой-то позиции i выполняется условие p[i] = |pattern|, то в этой позиции начинается очередное вхождение образца в цепочку.
\subsubsection{Код программы}
\begin{lstlisting}[language={Matlab}, caption={Алгоритм Рабина-Карпа}, label={Script}]
function [index] = KnuthMorrisPratt(str, pattern)

n = length(str);
m = length(pattern);

a = [pattern, '#', str]; 
pref = PrefixFuction(a);

for i = 1 : n
  if pref(m + i) == m
      index = i - m;
    return;
  end
end
end % End of 'KnuthMorrisPratt' function

function pref = PrefixFuction(str)
n = length(str);
pref = zeros(1, n);
for i = 2 : n
  k = pref(i - 1);
  while k > 0 && str(i) ~= str(k + 1)
    k = pref(k);
  end
  if str(i) == str(k + 1)
    k = k + 1;
  end
  pref(i) = k;
end
end % End of 'PrefixFuction' function
\end{lstlisting}
\subsubsection{Сложность алгоритма}
O(n + m)
\subsubsection{Результаты работы}
m = [10, 30, 70] (красный, зеленый, синий)
\begin{figure}[h]
	\centering
	\includegraphics[width=0.8\linewidth]{KMP.jpg}
	\caption{Результат работы алгоритма Кнута-Морриса-Пратта}
\end{figure}
\vfill
\section{Вывод}
Мы сравнили все 3 алгоритма. Тестовый код:
\begin{lstlisting}[language={Matlab}, caption={Тестовый скрипт}, label={Script}]
n = 100 : 100 : 10000;
m = 5 : 5 : 20;
t1 = zeros(length(m), length(n));
t2 = zeros(length(m), length(n));
t3 = zeros(length(m), length(n));
array = 'abcdefghijklmnopqrstuvwxyz';

for i = 1 : length(m)
  for j = 1 : length(n)
    for k = 1 : n(j)
      str(k) = array(floor(length(array) * rand(1, 1)) + 1);
    end

    index = floor((n(j) - m(i)) * rand(1, 1)) + 1;
    substr = str(index : index + m(i) - 1);

    tic
    resIndex = Pos(str, substr);
    t1(i, j) = t1(i, j) + toc;

    if (resIndex ~= index)
      display("Wrong result.");
    end

    tic
    resIndex = RabinKarp(str, substr);
    t2(i, j) = t2(i, j) + toc;

    if (resIndex ~= index)
      display("Wrong result.");
    end

    tic
    resIndex = KnuthMorrisPratt(str, substr);
    t3(i, j) = t3(i, j) + toc;

    if (resIndex ~= index)
      display("Wrong result.");
    end
  end
end
figure;
hold on;

grid on;
title('Substring search graphic'); 
xlabel('String length');
ylabel('Search time (seconds)');


for i = 1 : length(n)
  tmp(i) = t1(2, i);
end
plot(n, tmp, 'r')

for i = 1 : length(n)
  tmp(i) = t2(2, i);
end
plot(n, tmp, 'g')

for i = 1 : length(n)
  tmp(i) = t3(2, i);
end
plot(n, tmp, 'b')
\end{lstlisting}
Проведем сравнение при m = 10. Результаты представлены ниже (красный - \textbf{примитивный алгоритм}, зеленый - \textbf{алгоритм Рабина-Карпа}, красный - \textbf{алгоритм Кнута-Морриса-Пратта}):
\begin{figure}[h]
	\centering
	\includegraphics[width=0.8\linewidth]{Result.jpg}
	\caption{Результат сравнения всех трех алгоритмов}
\end{figure}

Результат сравнения скоростей алгоритмов:
\begin{enumerate}
	\item \textbf{Алгоритм Кнута-Морриса-Пратта}
	\item \textbf{Алгоритм Рабина-Карпа}
	\item \textbf{Примитивный алгоритм}
\end{enumerate}
\section{Библиографический список}
\begin{enumerate}
	\item https://neerc.ifmo.ru
	\item https://ru.wikipedia.org/
	\item Дж. Макконелл "Анализ алгоритмов"
\end{enumerate}
\end{document}