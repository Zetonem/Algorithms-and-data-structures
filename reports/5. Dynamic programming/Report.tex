\documentclass{article}
\usepackage[utf8]{inputenc}
\usepackage[russian]{babel}
\usepackage[14pt]{extsizes}
\usepackage[left=2.5cm,right=1cm,
top=2cm,bottom=2cm,bindingoffset=0cm]{geometry}
\usepackage{amsmath,amsfonts, amssymb}
\usepackage{graphicx}
\usepackage{listings}
\usepackage{xcolor, lipsum}
\lstloadlanguages{MATLAB}

\definecolor{back}{HTML}{FBFDFF}
\definecolor{board}{RGB}{77, 101, 104}
\definecolor{keywords}{RGB}{166, 38, 164}

\lstset{
	language=Matlab,
	commentstyle=\color[rgb]{0.8, 0.5, 0.3},
	showstringspaces=false,
	keywordstyle=\color{keywords},
	identifierstyle=\color{black},
	backgroundcolor=\color{back},
	extendedchars=\true,
	keepspaces=true,
}

\usepackage{caption}
\DeclareCaptionFont{white}{\color{white}}
\DeclareCaptionFormat{listing}{\colorbox{board}{\parbox{\textwidth}{\hspace{15pt}#1#2#3}}}
\captionsetup[lstlisting]{format=listing,labelfont=white,textfont=white, singlelinecheck=false, margin=0pt, font={bf,footnotesize}}

\begin{document}
\begin{titlepage}
  \begin{center}
    \large {
    ФГБОУ ВПО\\
    ПЕТЕРБУРГСКИЙ\\
    ГОСУДАРСТВЕННЫЙ УНИВЕРСИТЕТ ПУТЕЙ СООБЩЕНИЯ\\
    ИМПЕРАТОРА АЛЕКСАНДРА I}
     
    \vspace{2cm}
 
    Кафедра "Информационные и вычислительные системы"\\
    \bigskip
    \bigskip
    \bigskip
    \bigskip          
     
    \large
    ОТЧЕТ\\
    по лабораторной работе №2\\
    {\LARGE\textbf{Динамическое программирование}}
  \end{center}
 
\newlength{\ML}
\settowidth{\ML}{дент}
\bigskip
\bigskip
\bigskip
\bigskip
\bigskip
\bigskip

\begin{flushleft}
	Выполнил студент \hspace{4.0cm} $\underset{\text{(подпись)}}{\underline{\hspace*{2.5cm}}}$ \hfill Зайцев Л.А.\\
	Группа ИВБ-811
\end{flushleft}

\bigskip
\begin{flushleft}
	Отчет принял \hspace{\ML}\hspace{4.0cm} $\underset{\text{(подпись)}}{\underline{\hspace*{2.5cm}}}$ \hfill Баушев А.Н.
\end{flushleft}

\vfill
\begin{center}
  Санкт-Петербург, 2020 г.
\end{center}
\end{titlepage}
\newpage
\section{Алгоритма поиска наибольшей возрастающей подпоследовательности}
\subsection{Постановка задачи}
Дан массив из n чисел: a[1 \ldots n]. Требуется найти в этой последовательности строго возрастающую подпоследовательность наибольшей длины.
\subsection{Описание алгоритма}
Возьмем массив d[1 \ldots n], где d[i] - это длина наибольшей возрастающей подпоследовательности, оканчивающейся именно в элементе с индексом i.\\
Пусть текущий индекс — i, т.е. мы хотим посчитать значение d[i], а все предыдущие значения d[1] 
\ldots d[i-1] уже подсчитаны. Тогда заметим, что у нас есть два варианта:
\begin{itemize}
	\item d[i] = 1, т.е. искомая подпоследовательность состоит только из числа a[i].
	\item  d[i] > 1. Тогда перед числом a[i] в искомой подпоследовательности стоит какое-то другое число. Давайте переберём это число: это может быть любой элемент a[j] (j = 1 \ldots i-1), но такой, что a[j] < a[i]. Пусть мы рассматриваем какой-то текущий индекс j. Поскольку динамика d[j] для него уже подсчитана, получается, что это число a[j] вместе с числом a[i] даёт ответ d[j] + 1. Таким образом, d[i] можно считать по такой формуле:
	\begin{equation}
	d[i] = \underset{\underset{a[j] < a[i]}{j = 1 \ldots i - 1}}{max} (d[j] + 1)
	\end{equation}
\end{itemize}
Объединяя эти два варианта в один, получаем окончательный алгоритм для вычисления d[i]:
	\begin{equation}
	d[i] = max(1, \underset{\underset{a[j] < a[i]}{j = 1 \ldots i - 1}}{max} (d[j] + 1))
	\end{equation}
\subsection{Код программы}
\begin{lstlisting}[language={Matlab}, caption={Наибольшая возрастающая подпоследовательность}, label={Script}]
function [Res, Path] = LIS(array)
Res = 0;
Path = [];

n = length(array);
d = zeros(1, n);
prev = zeros(1, n);

for i = 1 : n
	d(i) = 1;
	prev(i) = -1;
	for j = 1 : n
		if array(j) < array(i) && 1 + d(j) > d(i)
			d(i) = d(j) + 1;
			prev(i) = j;
		end
	end
end

Res = d(1);
pos = 1;

for i = 1 : n
	if (d(i) > Res)
		Res = d(i);
		pos = i;
	end
end

i = 1;
while pos ~= -1
	Path(i) = pos;
	pos = prev(pos);
	i = i + 1;
end

n = length(Path);
for i = 1 : n / 2
	tmp = Path(i);
	Path(i) = Path(n - i + 1);
	Path(n - i + 1) = tmp;
end
end % End of 'LIS' function
\end{lstlisting}
\subsection{Сложность алгоритма}
O($n^{2}$)
\subsection{Результат работы}
\begin{lstlisting}[language={Matlab}, caption={Тест. Наибольшая возрастающая подпоследовательность}, label={Script}]
a(1) = 1;
a(2) = 5;
a(3) = 3;
a(4) = 7;
a(5) = 1;
a(6) = 4;
a(7) = 10;
a(8) = 15;

[ans, path] = LIS(a)
n = 100 : 100 : 10000
t1 = zeros(1, length(n));

for i = 1 : length(n)
	a = zeros(1, n(i));
	for k = 1 : n(i)
		a(k) = length(a) * rand(1, 1);
	end
	tic
	LIS(a);
	t1(i) = toc;
end
figure;
hold on;
plot(n, t1);
\end{lstlisting}
\begin{figure}[h]
	\centering
	\includegraphics[width=0.3\linewidth]{LIS.png}
	\caption{Результат работы алгоритма по поиску наибольшей возрастающей подпоследовательности}
\end{figure}
\begin{figure}[h]
	\centering
	\includegraphics[width=0.6\linewidth]{LIS2.png}
	\caption{Время работы алгоритма по поиску наибольшей возрастающей подпоследовательности}
\end{figure}
\section{Алгоритм поиска расстояния редактирования}
\subsection{Постановка задачи}
\textbf{Расстояние редактирования} - это минимальное количество операций вставки одного символа, удаления одного символа и замены одного символа на другой, необходимых для превращения одной строки в другую.
\subsection{Описание алгоритма}
Возьмем массив d[1 \ldots n][1 \ldots m], где m и n - длины строк ($S_{1}$ и $S_{2}$), а d[i][j] - расстояние между префиксами строк: первыми i символами строки $S_{1}$ и первыми j символами строки $S_{2}$. Оно вычисляется по формуле:
\begin{equation}
d[i][j] = 
\begin{cases}
0 & i = 1, j = 1;\\
i & j = 1, i > 1\\
j & j > 1, i = 1\\
d(i - 1, j - 1) & S_{1}[i] = S_{2}[j]\\
min(\\d(i, j - 1) + 1,\\ d(i - 1, j) + 1,\\ d(i - 1, j - 1) + 1) & j > 1, i > 1, S_{1}[i] \neq S_{2}[j]
\end{cases}
\end{equation}
\subsection{Код программы}
\begin{lstlisting}[language={Matlab}, caption={Расстояние редактирования}, label={Script}]
function [Res] = LevDist(a, b)
n = length(a) + 1;
m = length(b) + 1;
Res = 0;
d = zeros(n, m);

for i = 2 : n
	d(i, 1) = i - 1;
end

for j = 2 : m
	d(1, j) = j - 1;
end

for i = 2 : n
	for j = 2 : m
		if a(i - 1) ~= b(j - 1)
			d(i, j) = min(d(i, j - 1) + 1, min(d(i - 1, j) + 1, d(i - 1, j - 1) + 1));
		else
			d(i, j) = d(i - 1, j - 1);
		end
	end
end

d
Res = d(n, m);
end % End of 'LevDist' function
\end{lstlisting}
\subsection{Сложность алгоритма}
O(mn)
\subsection{Результат работы}
\begin{lstlisting}[language={Matlab}, caption={Тест. Расстояние редактирования}, label={Script}]
LevDist('exponential', 'polynomial')
\end{lstlisting}
\begin{figure}[h]
	\centering
	\includegraphics[width=0.75\linewidth]{LevDist.png}
	\caption{Результат работы алгоритма по поиску расстояния редактирования}
\end{figure}
\section{Задача о рюкзаке 0-1}
\subsection{Постановка задачи}
Дано N предметов, W — вместимость рюкзака, w = {$w_{1}, w_{2},$ \ldots , $w_{N}$} — соответствующий ему набор положительных целых весов, p = {$p_{1}, p_{2},$ \ldots , $p_{N}$} — соответствующий ему набор положительных целых стоимостей. Нужно найти набор бинарных величин B={$b_{1}, b_{2}$, \ldots , $b_{N}$}, где $b_{i}$ = 1, если предмет $n_{i}$ включен в набор, $b_{i}$ = 0, если предмет $n_{i}$ не включен, и такой что:
\begin{enumerate}
	\item $b_{1}w_{1}+ \ldots + b_{N}w_{N} \leq W$
	\item $b_{1}p_{1}+\ldots+b_{N}p_{N}$ максимальна.
\end{enumerate}
\subsection{Описание алгоритма}
Пусть A(k,s) есть максимальная стоимость предметов, которые можно уложить в рюкзак вместимости s, если можно использовать только первые k предметов, то есть ${n_{1},n_{2}, \ldots ,n_{k}}$, назовем этот набор допустимых предметов для A(k,s).\\

A(k,1) = 0\\

A(1,s) = 0\\

Найдем A(k,s). Возможны 2 варианта:

\begin{enumerate}
	\item Если предмет k не попал в рюкзак. Тогда A(k,s) равно максимальной стоимости рюкзака с такой же вместимостью и набором допустимых предметов ${n_{1}, \ldots , n_{k - 1}}$, то есть A(k, s) = A(k - 1, s)
	\item Если k попал в рюкзак. Тогда A(k,s) равно максимальной стоимости рюкзака, где вес s уменьшаем на вес k-ого предмета и набор допустимых предметов ${n_{1}, n_{2}, \ldots , n_{k - 1}}$ плюс стоимость k, то есть $A(k - 1, s - w_{k}) + p_{k}$
\end{enumerate}
\subsection{Код программы}
\begin{lstlisting}[language={Matlab}, caption={Тест. Расстояние редактирования}, label={Script}]
function [Res, Collection] = Knapsack01(w, p, Weight)
Collection = [];
Res = 0;
n = length(w);
A = zeros(n + 1, Weight + 1);

for i = 2 : n + 1
	for j = 1 : Weight + 1
		if j > w(i - 1)
			A(i, j) = max(A(i - 1, j), A(i - 1, j - w(i - 1)) + p(i - 1));
		else
			A(i, j) = A(i - 1, j);
		end
	end
end

A
Res = A(n + 1, Weight + 1);

i = n + 1;
j = Weight + 1;
index = 1;

while A(i, j) ~= 0
	if A(i - 1, j) == A(i, j)
		i = i - 1;
	else
		i = i - 1;
		j = j - w(i);
		Collection(index) = i;
		index = index + 1;
	end
end
end % End of 'Knapsack01' function
\end{lstlisting}
\subsection{Сложность алгоритма}
O(nW)
\subsection{Результат работы}
\begin{lstlisting}[language={Matlab}, caption={Тест. Задача о рюкзаке без повторений}, label={Script}]
Knapsack01([3, 4, 5, 8, 9], [1, 6, 4, 7, 6], 13)
\end{lstlisting}
\begin{figure}[h]
	\centering
	\includegraphics[width=0.75\linewidth]{Knapsack01.png}
	\caption{Результат работы алгоритма задача о рюкзаке без повторений}
\end{figure}
\section{Задача о неограниченном рюкзаке}
\subsection{Постановка задачи}
Теперь любой предмет можно брать неограниченное количество раз.
\subsection{Описание алгоритма}
Пусть d(i,j) - максимальная стоимость любого количества вещей типов от 1 до i, суммарным весом до j включительно.
Заполним d(0,j) нулями.
Тогда меняя i от 2 до N, рассчитаем на каждом шаге d(i, j), для c от 1 до W, по рекуррентной формуле:
\begin{equation}
d(i, j)=\begin{cases}
d(i - 1, j) & for j = 0, \ldots , w_{i} - 1\\
max(d(i - 1, j), d(i, j - w_{i}) + p_{i}) & for j = w_{i}, \ldots , W;
\end{cases}
\end{equation}
\subsection{Код программы}
\begin{lstlisting}[language={Matlab}, caption={Тест. Задача о неограниченном рюкзаке}, label={Script}]
function [Res, Collection] = Knapsack(w, p, Weight)
Res = 0;
Collection = [];

n = length(w);
d = zeros(n + 1, Weight + 1);

for i = 2 : n + 1
	for j = 1 : Weight + 1
		if j > w(i - 1)
			d(i, j) = max(d(i - 1, j), d(i, j - w(i - 1)) + p(i - 1));
		else
			d(i, j) = d(i - 1, j);
		end
	end
end
d
Res = d(i, j);

i = n + 1;
j = Weight + 1;
index = 1;

while d(i, j) ~= 0
	if d(i - 1, j) == d(i, j)
		i = i - 1;
	else
		j = j - w(i - 1);
		Collection(index) = i - 1;
		index = index + 1;
	end
end
end % End of 'Knapsack' function
\end{lstlisting}

\subsection{Сложность алгоритма}
O(nW)
\subsection{Результат работы}
\begin{lstlisting}[language={Matlab}, caption={Тест. Задача о неограниченном рюкзаке}, label={Script}]
[ans, coll] = Knapsack([3, 4, 5, 8, 9], [1, 6, 4, 7, 6], 13)
\end{lstlisting}
\newpage
\begin{figure}[h!]
	\centering
	\includegraphics[width=0.75\linewidth]{Knapsack.png}
	\caption{Результат работы алгоритма задача о неограниченном рюкзаке}
\end{figure}
\section{Вывод}
В ходе выполнения лабораторной работы мы познакомились с методом динамического программирования путем решения классических задач. Динамическое программирование - способ решения сложных задач путём разбиения их на более простые подзадачи. Мы научились составлять функциональные уравнения динамического программирования.
\section{Библиографический список}
\begin{enumerate}
	\item https://ru.wikipedia.org/
	\item https://neerc.ifmo.ru/
\end{enumerate}
\end{document}